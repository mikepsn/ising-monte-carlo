%========================================================================
% COMPUTATIONAL PHYSICS - PROJECT 4
% ISING MONTE-CARLO
%========================================================================

\documentclass[a4paper]{IEEEtran} 

\usepackage{amssymb}
\usepackage{moreverb}
\usepackage[cmex10]{amsmath} 
\usepackage{cite} 
\usepackage{graphicx} 
\usepackage[colorlinks=false, hidelinks]{hyperref} 

\usepackage{listings} 
\usepackage{color} 
\usepackage{xcolor}  
\usepackage{microtype} 
\usepackage{microtype} 
\usepackage{inconsolata} 
\usepackage[framemethod=TikZ]{mdframed} 
\usepackage{alltt}
\usepackage{sverb} 
\usepackage{verbatim} 
\usepackage{pifont} 
\usepackage{alltt} 
\usepackage{helvet} 

%==============================================================================

\newcommand{\sz}{\underline{\sigma}^z}
\newcommand{\sg}[2]{\sigma_{#1,#2}}
\newcommand{\av}[1]{\langle #1 \rangle}

\title{Monte-Carlo Study of the 2D Ising Model}
\author{Michael Papasimeon\\ August 25, 1997} 
\date{August 25, 1997} 

%==============================================================================
%----------------------------------------------------------------------
% CODE LISTING SETTINGS
%----------------------------------------------------------------------

\lstset{language=fortran,
        %basicstyle=\footnotesize\ttfamily, 
        basicstyle=\small\ttfamily,
        columns=fullflexible, 
        %title=\lstname, 
        numbers=left, stringstyle=\texttt, 
        numberstyle={\tiny\texttt}, 
        keywordstyle=\color{blue}, 
        commentstyle=\color{darkgreen}, 
        stringstyle=\color{purple} } 


\mdfsetup{skipabove=\topskip, skipbelow=\topskip} 

\definecolor{codebg}{rgb}{0.99,0.99,0.99}

\global\mdfdefinestyle{code}{%
    frametitlerule=true,%
    frametitlefont=\small\bfseries\ttfamily,%
    frametitlebackgroundcolor=lightgray,%
    backgroundcolor=codebg,%
    linecolor=gray, linewidth=0.5pt,%
    leftmargin=0.5cm, rightmargin=0.5cm,%
    roundcorner=2pt,%
    innerleftmargin=5pt
}

\global\mdfdefinestyle{code2}{%
    topline=false,%
    bottomline=false,%
    leftline=true,%
    rightline=false,%
    backgroundcolor=codebg,%
    linecolor=gray, linewidth=0.5pt,%
    leftmargin=0.0cm, rightmargin=0.0cm,%
    innerleftmargin=1pt
}

\newcommand{\showcode}[1]{\begin{mdframed}[style=code] %
                            \lstinputlisting{#1}% 
                          \end{mdframed}% 
}

\newcommand{\showsmallcode}[1]{\begin{mdframed}[style=code2] %
        \lstinputlisting[basicstyle=\ttfamily\tiny]{#1}% 
                          \end{mdframed}% 
}


%----------------------------------------------------------------------
% IEEE SETTINGS
%----------------------------------------------------------------------

\interdisplaylinepenalty=2500
\setlength{\IEEEilabelindent}{\IEEEilabelindentB}

%\markboth{630--364 Computational Physics}{} 
\markboth{Monte-Carlo Study of the 2D Ising Model. Michael Papasimeon. 1997}{} 


%==============================================================================

\begin{document}
\maketitle

%==============================================================================

\begin{abstract}
This paper was written for an introductory undegraduate class in computational
physics in 1997. It focuses on Monte-Carlo simulation of the 2D Ising model.
The work models a regular grid of particles each with spin up or down. The work
attempts to determine the critical temperature $\beta_c$ where a phase
transition occurs. 
The source code listings in \textsc{Fortran77} can be found in the appendices.
\end{abstract}

\section{Aims}
The main aims are to:
\begin{itemize}
    \item Write a FORTRAN program which can represent 
          a 2D lattice of spin up/down particles using the 2D Ising model.
    \item Extend the program to use a Monte-Carlo technique to investigate
          the magnetisation of the lattice for different temperatures.
    \item Use the program to attempt to determine the critical temperature
          (where a phase transition occurs) for the ferro-magnetic Ising 
          model.
\end{itemize}

%==============================================================================

\section{Theory}
    Figure~\ref{fig:grid} represents a two dimensional lattice of 
    $N_s = N_x \times N_y$ particles. Each particle can be either 
    spin up or spin down. The spin at site $\alpha = (i,j)$
    is given by $\sigma^{z}_{\alpha}$. 
    A lattice of $N_s$ particles can be in $2^{N_s}$ possible configurations,
    represented as $\sz = ( \sigma^z_1,...., \sigma^z_{N_s} )$.
    The total energy for a given configuration is given by:
    \[ E(\sigma^z) = -J \sum_{\langle \alpha \beta \rangle} 
                        \sigma_{\alpha}^{z} \sigma_{\beta}^{z}
                   -B_z \sum_{\alpha } \sigma_{\alpha}^{z} \]

    For this experiment we take $B_z = 0$ because there is no
    external magnetic field and $J = 1$, because we are considering
    the case of a ferromagnet ($J > 0$), to get:
    \[ E(\sigma^z) = -\sum_{\langle \alpha \beta \rangle} 
                      \sigma_{\alpha}^{z} \sigma_{\beta}^{z} \]
    We can determine the partition function of the system because we know
    from statistical mechanics that each configuration is a micro-state of
    the canonical ensemble. The partition function is then given by
    \[ Z = \sum_{\sz } e^{-\beta \sz } \] .
    The probability of obtaining a particular state is given by
    \[ P(\sz) = \frac{e^{-\beta E(\sz) }}{Z} \]
    The ensemble average of a physical observable property is:
    \[ X_{obs} = \langle X \rangle = \sum_{\sz} X(\sz) P(\sz) \]
    Therefore the magnetisation of the entire lattice is given
    by summing all the spins at all lattice sites.
    \[ M(\sz) = \sum_{N_s}^{\alpha = 1} \sigma_{\alpha} \]

    \begin{figure}[t]
        \caption{2D lattice of $N_x \times N_y$ particles. }
        \label{fig:grid}
        \begin{center}
            \includegraphics[angle=-90,width=7cm]{lattice.eps}
        \end{center}
    \end{figure} 
    
    We can obtain a number of configurations and determine their magnetisation
    using the Metropolis algorithm. This involves sweeping through the lattice
    and selecting a number of sites at random. The magnetisation can then
    be calculated to be approximated to be the Monte-Carlo average given
    by:
    \[ M = \av{M} \approx \frac{1}{N} \sum^{N}_{k=1} M_k \]
    The weight function used in the Monte-Carlo algorithm is the Boltzmann
    distribution.
    At the randomly selected sites $(i,j)$
    \begin{verbatim}
        flip_spin(i,j)
        if (change in energy < 0)
            accept
        else
            if (exp(-beta * change in energy) > eta)
                accept
            else
                reject
    \end{verbatim}
    
    The energy of an entire lattice $E$ is the sum of the energies
    of all the particles in the lattice. The energy of a single particle
    is has contributions from it's nearest neighbours in the following
    way:
    \[ \epsilon = -J \sg{i}{j} \left[ \sg{i-1}{j} +
                                      \sg{i}{j+1} +
                                      \sg{i+1}{j} +
                                      \sg{i}{j-1} \right] \]

%==============================================================================

\section{Method}
    The following method was followed in this experiment:
    \begin{itemize}
        \item A FORTRAN program (\texttt{ising.f} in Appendix A), was
              written which represented a 2D lattice of spin up/down 
              particles.
            \begin{itemize}
                \item The user of the program has the option of
                      starting the lattice with a cold start
                      (all spins up) or a hot start with all 
                      spins randomly selected to be up or down
                      using the pseudo-random number 
                      generating function \texttt{rand()} 
                \item A function to print the lattice on the screen
                      was written
            \end{itemize}
        \item A function to calculate the energy of a lattice was
              written making use of periodic boundary conditions.
        \item A function to determine the the magnetisation of 
              a particular lattice configuration was written.
        \item A function to apply the Metropolis algorithm to a given
              lattice, $\beta$ and for user specified number of sweeps.
              The function allows 100 sweeps to allow the lattice
              to thermalise.
              The magnetisation is and standard deviation are calculated
              over 100 configurations, with each configuration separated 
              by 10 sweeps.
        \item The Metropolis function is called a number of times
              for different values of $\beta$ ranging from $0.1$ 
              to $1.0$.
    \end{itemize}

%==============================================================================

\section{Results}

    Results from the simulation are shown in Figures~\ref{fig:4x4-1} 
    to~\ref{fig:16x16-2}. 
    The figures plot the absolute value of the 
    magnetisation $|M|$ against the inverse of the
    temperature $\beta$.

    The program was run for a number of different lattices sizes.
    The lattices sizes used are: $4 \times 4$, 
    $6 \times 6$, $8 \times 8$, $10 \times 10$, $12 \times 12$,
    $14 \times 14$ and $16 \times 16$.

    There are two sets of plots. 
    Set 1 (Figures~\ref{fig:4x4-1} to~\ref{fig:16x16-1})
    contains plots of the magnetisation
    against the inverse temperature over 100 configurations
    and 10 sweeps per configuration. 
    Set 2 (Figures~\ref{fig:4x4-2} to~\ref{fig:16x16-2})
    contains plots over 50
    configurations with 40 sweeps per configuration.

    In all the plots the error bars show the error in the magnetisation,
    where we have:
    \[ M = \av{M} \pm \sigma_M \]
    The error bars are the standard deviation in $M$, $\sigma_M$. 
    The variance $\sigma_{M}^2$ is given by:
    \[ \sigma_{M}^2 = \frac{1}{N} \left( \av{M^2} - \av{M}^2 \right) \]

    Using the graphs, we can determine the temperature at which
    the ferromagnetic phase transition occurs, the critical temperature $\beta_c$.
    Table~\ref{tbl:betac} summarises these results.

    \begin{table}[ht]
        \begin{center} 
        \caption{Approximate Estimates for Critical Temperature
                 $\beta_c$ from Magnetisation plots.}
        \label{tbl:betac} 

        \begin{tabular}{|r|r|c|c|} \hline
        $N_x \times N_y$ & $N_s$ & Approx $\beta_c$ (Set 1)
                                 & Approx $\beta_c$ (Set 2) \\ \hline
        4x4   & 16  & 0.2 & 0.18 \\ \hline
        6x6   & 36  & 0.2 & 0.2  \\ \hline
        8x8   & 64  & 0.2 & 0.2  \\ \hline
        10x10 & 100 & 0.2 & 0.2  \\ \hline
        12x12 & 144 & 0.2 & 0.15 \\ \hline
        14x14 & 196 & 0.21 & 0.2 \\ \hline
        16x16 & 256 & 0.3 & 0.2  \\ \hline
        \end{tabular}
        \end{center} 
    \end{table} 

    \section{Discussion}

    In all the simulations of different sized lattices, it is
    clear to see in the plots of the magnetisation that there is a phase
    transition occurring.

    What is not clear however is the exact critical temperature $\beta_c$.
    In most of the cases (for both sweep/configuration setups) the
    critical temperature is approximately $\beta_c \approx 0.2$.
    In the large lattice limit we expect to get a value for the critical
    temperature of $\beta_c = 0.4407$. The only result which approaches
    this is that of Set 1 for a $16 \times 16$ lattice. For the case
    of the $8 \times 8$ lattice we obtain $\beta_c \approx 0.2$.
    Without trying
    larger lattice sizes we are unable to tell if this is realistic
    or merely a statistical aberration.

    The magnetisation in the temperature ranging from $\beta = 0.1$ to 
    the critical temperature $\beta_c$, tends to be quite erratic
    in all the simulations, with large errors in a number of cases.
    This made determining the critical temperature difficult, and
    therefore only approximate results could be obtained.

    The question of finding a relationship between the total lattice
    size $N_s$ and the critical temperature $\beta_c$ cannot be answered
    because the majority of the results indicate that $\beta_c \approx 0.2$
    for $16 \le N_s \le 256 $.
    Due to the statistical nature of this simulation, it would be more
    realistic to go through many runs for each lattice size and then
    take the mean of the estimated critical temperatures.




%==============================================================================

    \section{Plots of Magnetisation as a function of Inverse Temperature} 
    This appendix contains the results from all the simulations conducted.
    The figures show inverse temperature $\beta$ on the $x$-axis and the 
    grid magnetisation $|M|$ on the $y$-axis. The figures show simulations of 
    various grid sizes ranging from a small $4 \times 4$ grid all the way up
    to a $16 \times 16$ grid. The results are divided into two sets.

    Set 1 (Figures~\ref{fig:4x4-1} to~\ref{fig:16x16-1})
    contains plots of the magnetisation
    against the inverse temperature over 100 configurations
    and 10 sweeps per configuration. 

    Set 2 (Figures~\ref{fig:4x4-2} to~\ref{fig:16x16-2})
    contains plots over 50
    configurations with 40 sweeps per configuration.


    \subsection{Set 1: 100 configurations with 10 sweeps/configuration} 

    
    \begin{figure}[h] 
    \caption{Lattice Size $\mathbf{4 \times 4}$}
    \label{fig:4x4-1}
    \begin{center}
        \includegraphics[width=0.99\columnwidth]{4x4_1.eps}
    \end{center}
    \end{figure} 

    \begin{figure}
    \caption{Lattice Size $\mathbf{6 \times 6}$}
    \label{fig:6x6-1}
    \begin{center}
        \includegraphics[width=0.99\columnwidth]{6x6_1.eps}
    \end{center}
    \end{figure} 

    \begin{figure}
    \caption{Lattice Size $\mathbf{8 \times 8}$}
    \label{fig:8x8-1} 
    \begin{center}
        \includegraphics[width=0.99\columnwidth]{8x8_1.eps}
    \end{center}
    \end{figure} 

    \begin{figure} 
    \caption{Lattice Size $\mathbf{10 \times 10}$}
    \label{fig:10x10-1} 
    \begin{center}
        \includegraphics[width=0.99\columnwidth]{10x10_1.eps}
    \end{center}
    \end{figure} 

    \begin{figure} 
    \caption{Lattice Size $\mathbf{12 \times 12}$}
    \label{fig:12x12-1} 
    \begin{center}
        \includegraphics[width=0.99\columnwidth]{12x12_1.eps}
    \end{center}
    \end{figure} 

    \begin{figure} 
    \caption{Lattice Size $\mathbf{14 \times 14}$}
    \label{fig:14x14-1}
    \begin{center}
        \includegraphics[width=0.99\columnwidth]{14x14_1.eps}
    \end{center}
    \end{figure} 

    \begin{figure} 
    \caption{Lattice Size $\mathbf{16 \times 16}$}
    \label{fig:16x16-1} 
    \begin{center}
        \includegraphics[width=0.99\columnwidth]{16x16_1.eps}
    \end{center}
    \end{figure} 

%==============================================================================

    \subsection{Set 2:  50 Configurations and 40 Sweeps/Configuration}
   
    \newpage
    \begin{figure}
    \caption{Lattice Size $\mathbf{4 \times 4}$}
    \label{fig:4x4-2} 
    \begin{center} 
        \includegraphics[width=0.99\columnwidth]{4x4_2.eps}
    \end{center}
    \end{figure} 

    \begin{figure} 
    \caption{Lattice Size $\mathbf{6 \times 6}$}
    \label{fig:6x6-2} 
    \begin{center}
        \includegraphics[width=0.99\columnwidth]{6x6_2.eps}
    \end{center}
    \end{figure} 

    \begin{figure} 
    \caption{Lattice Size $\mathbf{8 \times 8}$}
    \label{fig:8x8-2} 
    \begin{center}
        \includegraphics[width=0.99\columnwidth]{8x8_2.eps}
    \end{center}
    \end{figure} 

    \begin{figure} 
    \caption{Lattice Size $\mathbf{10 \times 10}$}
    \label{fig:10x10-2} 
    \begin{center}
        \includegraphics[width=0.99\columnwidth]{10x10_2.eps}
    \end{center}
    \end{figure} 

    \begin{figure}
    \caption{Lattice Size $\mathbf{12 \times 12}$}
    \label{fig:12x12-2} 
    \begin{center}
        \includegraphics[width=0.99\columnwidth]{12x12_2.eps}
    \end{center}
    \end{figure} 

    \begin{figure} 
    \caption{Lattice Size $\mathbf{14 \times 14}$}
    \label{fig:14x14-2} 
    \begin{center}
        \includegraphics[width=0.99\columnwidth]{14x14_2.eps}
    \end{center}
    \end{figure} 

    \begin{figure} 
    \caption{Lattice Size $\mathbf{16 \times 16}$}
    \label{fig:16x16-2} 
    \begin{center}
        \includegraphics[width=0.99\columnwidth]{16x16_2.eps}
    \end{center}
    \end{figure} 

%==============================================================================



%==============================================================================

    \onecolumn
    \appendix[Code Listing: ising.f]
    \showcode{ising.f} 

\end{document}

%==============================================================================



